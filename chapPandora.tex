\chapter{Reconstruction}
\label{chap:Reconstruction}

\chapterquote{How to open a pandora box?}%
{A wise Chinese}%: Blackwood's Magazine May 1830


\section{Reconstruction overall}
digitisation
tracking

\section{Pandora}

Track quality cuts

\subsection{Cone clustering}
\label{sec:pandoraConeClustering}

% ATTN add a cone cluster picture?

Cone clustering is used in \pandora for grouping calorimeter hits, within a opening angle of the seed hit. Because the direction of particle flows is largely unchanged from the originated particle, whether it is a electromagnetic shower, QCD radiation or hadronisation, these cone clusters have similar direction and energy to the originated particle.

Typically a high energy calorimeter hit will be chosen as a ``seed''. A cone with a specified opening angle and depth will be formed around the seed. The \fourMomentum of calorimeter hits sum to the cone's \fourMomentum.

These cone clustering algorithms are widely used in the calorimeter in \pandora, and they produce basic working objects, \clusters.


% double counting taking care in pandora


Iterative track-cluster association

Photon, passage through matter

Muon ID

Fragmentation
