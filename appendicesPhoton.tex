%% The "\appendix" call has already been made in the declaration
%% of the "appendices" environment (see thesis.tex).
\chapter{Photon Reconstruction in \pandora}
\label{app:photon}

\chapterquote{I was an adventurer like you, then I took an arrow in the knee.}
{The town guard, Skyrim \cite{Skyrim}, 2011}

%Appendixes (or should that be ``appendices''?) make you look really clever, 'cos
%it's like you had more clever stuff to say than could be fitted into the main
%bit of your thesis. Yeah. So everyone should have at least three of them\dots

Here are extra tables for the \Chapter{chap:Photon}.



\begin{table}[htbp]
\centering

\smallskip

\begin{tabular}{l  r  r }
\hline
\hline
$E_f\leqslant1$\,GeV &  Photon$-$photon & Photon$-$neutral-hadron \\
\hline
\multicolumn{1}{L{0.3\textwidth}}{Transverse shower comparison, or} & \multicolumn{1}{R{0.3\textwidth}}{$d < 30 $\,mm; $\frac{E_{p1}}{E_m + E_f} > 0.9 $; $\frac{E_{p2}}{E_f} < 0.5 $; $E_{p1} > E_m$}  & \multicolumn{1}{R{0.3\textwidth}}{-} \\
%\multicolumn{1}{L{0.3\textwidth}}{close proximity, or} & \multicolumn{1}{R{0.3\textwidth}}{-}  & \multicolumn{1}{R{0.3\textwidth}}{$d < 20 $\,mm; $d_c < 40 $\,mm} \\
\multicolumn{1}{L{0.3\textwidth}}{Low energy fragment, or} & \multicolumn{1}{R{0.3\textwidth}}{$d < 20 $\,mm; $E_f < 0.4 $\,GeV}  & \multicolumn{1}{R{0.3\textwidth}}{$d < 20 $\,mm; $d_c < 40 $\,mm} \\
\multicolumn{1}{L{0.3\textwidth}}{Small fragment 1, or} & \multicolumn{1}{R{0.3\textwidth}}{$d < 30 $\,mm; $N_{calo} < 40 $; $d_c < 50 $\,mm}  & \multicolumn{1}{R{0.3\textwidth}}{$d < 50 $\,mm; $N_{calo} < 10 $; $d_h < 50$\,mm} \\
\multicolumn{1}{L{0.3\textwidth}}{Small fragment 2, or} & \multicolumn{1}{R{0.3\textwidth}}{$d < 50 $\,mm; $N_{calo} < 20 $}  & \multicolumn{1}{R{0.3\textwidth}}{-} \\
\multicolumn{1}{L{0.3\textwidth}}{Small fragment forward region, or} & \multicolumn{1}{R{0.3\textwidth}}{$N_{calo} < 40$; $d_c < 60$\,mm; $E_f < 0.6$\,GeV; $\absCosTheta > 0.7$}  & \multicolumn{1}{R{0.3\textwidth}}{-} \\
\multicolumn{1}{L{0.3\textwidth}}{Relative low energy fragment} & \multicolumn{1}{R{0.3\textwidth}}{$d < 40$\,mm; $d_h < 20$\,mm; $\frac{E_{f}}{E_m} < 0.01$}  & \multicolumn{1}{R{0.3\textwidth}}{$d < 40$\,mm; $d_h < 15$\,mm; $\frac{E_{f}}{E_m} < 0.01$} \\
\hline
$E_f>1$\,GeV &  Photon$-$photon & Photon$-$neutral-hadron \\
\hline
\multicolumn{1}{L{0.3\textwidth}}{Transverse shower comparison, or} & \multicolumn{1}{R{0.3\textwidth}}{$\frac{E_{p1}}{E_m + E_f} > 0.9 $; $E_{p2} = 0$ or ($\frac{E_{p2}}{E_f} < 0.5 $, $E_{p1} > E_m$)}  & \multicolumn{1}{R{0.3\textwidth}}{$\frac{E_{p1}}{E_m + E_f} > 0.9 $; $E_{p2} = 0$ or ($\frac{E_{p2}}{E_f} < 0.5 $, $E_{p1} > E_m$)} \\
\multicolumn{1}{L{0.3\textwidth}}{Relative low energy fragment 1, or} & \multicolumn{1}{R{0.3\textwidth}}{$d < 40$\,mm; $d_h < 20$\,mm; $\frac{E_f}{E_m} < 0.02$} & \multicolumn{1}{R{0.3\textwidth}}{$d < 40$\,mm; $d_h < 20$\,mm; $\frac{E_f}{E_m} < 0.02$} \\
\multicolumn{1}{L{0.3\textwidth}}{Relative low energy fragment 2, or} & \multicolumn{1}{R{0.3\textwidth}}{-}  & \multicolumn{1}{R{0.3\textwidth}}{$d < 40$\,mm; $d_h < 20$\,mm; $\frac{E_f}{E_m} < 0.1$; $E_f > 10$\,GeV} \\
\multicolumn{1}{L{0.3\textwidth}}{Relative low energy fragment 3} & \multicolumn{1}{R{0.3\textwidth}}{-}  & \multicolumn{1}{R{0.3\textwidth}}{$d < 20$\,mm; $d_h < 20$\,mm; $\frac{E_f}{E_m} < 0.2$; $E_f > 10$\,GeV} \\
\hline
\hline
\end{tabular}

\caption[The cuts for photon fragment removal algorithm in the \ECAL.]%
{The cuts for merging photon$-$photon-fragment pairs and photon$-$neutral-hadron-fragment pairs for both low energy and high energy fragments, after charged hadron reconstruction. Variables $d$, $d_c$ and $d_h$ are the mean energy weighted intra-layer distance of the pair, the distance between centroids, the minimum distance between calorimeter hits of the pair, respectively. Variables $E_m$ and $E_f$ are the main photon energy and the fragment energy, respectively. Variables $E_{p1}$ and $E_{p2}$ are the energies the two largest peaks, found by \peakFinding algorithm, ordered by descending energy, respectively. $N_{calo}$ is the number of the calorimeter hits in the fragment. $\absCosTheta$ is the absolute cosine of the polar angle, where beam direction is the z-axis. }
\label{tab:photonFragRemovalCuts}
\end{table}




\begin{table}[htbp]
\centering

\smallskip

\begin{tabular}{l  r  r }
\hline
\hline
$E_f\leqslant1$\,GeV &  Photon$-$photon & Photon$-$neutral-hadron \\
\hline
\multicolumn{1}{L{0.3\textwidth}}{Transverse shower comparison, or} & \multicolumn{1}{R{0.3\textwidth}}{$d < 20 $\,mm; $\frac{E_{p1}}{E_m + E_f} > 0.9 $; $E_{p2} = 0$ or ($\frac{E_{p2}}{E_f} < 0.5 $, $E_{p1} > E_m$)}  & \multicolumn{1}{R{0.3\textwidth}}{$d < 20 $\,mm; $\frac{E_{p1}}{E_m + E_f} > 0.9 $; $E_{p2} = 0$ or ($\frac{E_{p2}}{E_f} < 0.5 $, $E_{p1} > E_m$)} \\
\multicolumn{1}{L{0.3\textwidth}}{Low energy fragment, or} & \multicolumn{1}{R{0.3\textwidth}}{$d < 20 $\,mm; $E_f < 0.2 $\,GeV}  & \multicolumn{1}{R{0.3\textwidth}}{-} \\
\multicolumn{1}{L{0.3\textwidth}}{Small fragment 1, or} & \multicolumn{1}{R{0.3\textwidth}}{$d < 30 $\,mm; $N_{calo} < 20 $; $d_h < 13 $\,mm}  & \multicolumn{1}{R{0.3\textwidth}}{$d < 50 $\,mm; $N_{calo} < 10 $; $d_h < 50$\,mm} \\
\multicolumn{1}{L{0.3\textwidth}}{Small fragment 2, or} & \multicolumn{1}{R{0.3\textwidth}}{$d_c < 30 $\,mm; $N_{calo} < 10 $; $d_h < 13 $\,mm}  & \multicolumn{1}{R{0.3\textwidth}}{-} \\

\multicolumn{1}{L{0.3\textwidth}}{Relative low energy fragment} & \multicolumn{1}{R{0.3\textwidth}}{-}  & \multicolumn{1}{R{0.3\textwidth}}{$d < 40$\,mm; $d_h < 15$\,mm; $\frac{E_{f}}{E_m} < 0.01$} \\
\hline
$E_f>1$\,GeV &  Photon$-$photon & Photon$-$neutral-hadron \\
\hline
\multicolumn{1}{L{0.3\textwidth}}{Transverse shower comparison, or} & \multicolumn{1}{R{0.3\textwidth}}{$d< 20$\,mm; $\frac{E_{p1}}{E_m + E_f} > 0.9 $; $E_{p2} = 0$ or ($\frac{E_{p2}}{E_f} < 0.5 $, $E_{p1} > E_m$)}  & \multicolumn{1}{R{0.3\textwidth}}{$d< 20$\,mm; $\frac{E_{p1}}{E_m + E_f} > 0.9 $; $E_{p2} = 0$ or ($\frac{E_{p2}}{E_f} < 0.5 $, $E_{p1} > E_m$)} \\
\multicolumn{1}{L{0.3\textwidth}}{Relative low energy fragment } & \multicolumn{1}{R{0.3\textwidth}}{-} & \multicolumn{1}{R{0.3\textwidth}}{$d < 40$\,mm; $d_h < 20$\,mm; $\frac{E_f}{E_m} < 0.02$} \\
\hline
\hline
\end{tabular}

\caption[The cuts for photon fragment removal algorithm in the \ECAL.]%
{The cuts for merging photon$-$photon-fragment pairs and photon$-$neutral-hadron-fragment pairs for both low energy and high energy fragments, immediately after photon reconstruction. Variables $d$, $d_c$ and $d_h$ are the mean energy weighted intra-layer distance of the pair, the distance between centroids, the minimum distance between calorimeter hits of the pair, respectively. Variables $E_m$ and $E_f$ are the main photon energy and the fragment energy, respectively. Variables $E_{p1}$ and $E_{p2}$ are the energies the two largest peaks, found by \peakFinding algorithm, ordered by descending energy, respectively. $N_{calo}$ is the number of the calorimeter hits in the fragment.}
\label{tab:photonFragRemovalCuts2}
\end{table}