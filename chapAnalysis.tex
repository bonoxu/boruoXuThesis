\chapter{Analysis technique}
\label{chap:Reconstruction}

\chapterquote{In preparing for battle I have always found that plans are useless, but planning is indispensable.}%
{Dwight D. Eisenhower}%:

Automated analysis is the only way to deal with the vast amount of data generated in the high energy physics. In the last chapter we described the automated reconstruction tools in details. This chapter is dedicated to the common automated analysis tools and techniques, which will be used in the analysis described in subsequent chapters.

For the linear collider, thanks to the high granular calorimeter, the starting point for analysis would be individual Particle Flow Objects, as well as individual tracks. Each of the PFOs encodes four-momentum and position information. For tracks, they would have momentum and position information.

However, sometimes it is interesting to group PFOs and tracks into jets, where a jet is the result of hadronisation process from high energy particles like quarks or gulons.

\section{Jet algorithm}

A jet is typically a visually obvious structure in a event display. The momentum and the direction of a jet tend to resemble the originated particle. Despite the relative easiness of identifying jets visually, it presents a challenge for a pattern recognition program to identify jets effectively and efficiently.

Early work on jet finding started in 1977 \cite{PhysRevLett.39.1436}, where later development can be found in reviews \cite{Moretti:1998qx,Salam:2009jx,Ali:2010tw}.

There are two large families of jet finding algorithm, cone based algorithm, and sequential combination algorithm. Cone based algorithm is briefly discussed in \Section{sec:pandoraConeClustering}.

Sequential combination algorithm typically calculate a pair-wise distance metric. Pairs with the smallest metric will be combined. The metric will be calculated and updated, and a pair with smallest metric will be combined. This procedure will be repeated until some stopping criterion are satisfied.

The chosen jet algorithm implementation is FastJet C++ software package \cite{Cacciari:2011ma,Cacciari:2005hq}, providing a wide range of jet finding algorithms. The implementation in Marlin software package is called MarlinFastJet. The symbols in the subsequent discussion about specific jet algorithms will follow \cite{Cacciari:2011ma}

\subsection{\kt algorithm}

One of the common sequential combination algorithms for \pp collider experiment, is longitudinally-invariant \kt algorithm \cite{Catani:1993hr,Ellis:1993tq}. In the inclusive variant, The symmetrical pair-wise distance metric between particle $i$ and $j$, and the beam distance, are defined as
\begin{equation}
&d_{ij} = d_{ji} = \min\!\parenths{\pT_{i}^{2},\!\pT_{j}^{2}}\frac{\DeltaOf{R_{ij}^{2}}}{R^{2}}, \\
&d_{iB} = \pT_{i}^2,
\end{equation}
where $\pT_{i}$ is the transverse momentum of particle $i$ with respect to the beam ($z$) direction, and $\DeltaOf{R_{ij}^{2}}$ is the measurement of angular separation of particle $i$ and $j$. Formally $\DeltaOf{R_{ij}^{2}} = \parenths{y_i - y_j}^2 + \parenths{\phi_i - \phi_j}^2$, where $y_i = \frac{1}{2}\ln\!\frac{E_i + {p_z}_i}{E_i - {p_z}_i}$ and $\phi_i$ are particle $i$'s rapidity and azimuthal angle. R is a free parameter controlling the jet radius.

If $d_{ij} < d_{iB}$, particle $i$ and $j$ are merged, with the \fourMomentum of particle $i$ updated as the sum. Otherwise, particle $i$ is set to be a final jet, and delete from the particle list. The above procedure is repeated until no particle left.

The exclusive variant is similar. First difference is that when  $d_{iB} < d_{ij}$, the particle $i$ is discarded and part of the beam jet. The second difference is that when both $d_{ij}$ and $d_{iB}$ are above some threshold, $d_{cut}$, the clustering will stop. In practise, exclusive mode allows a specified number of jets to be found, which will automatically choose the $d_{cut}$. The inclusive mode would fine as many jets as the algorithm allows.

\subsection{Durham algorithm}

Durham algorithm \cite{Catani:1991hj}, also known as \ee \kt algorithm, is commonly used \ee collider experiment. It has a single distance metric:
\begin{equation}
d_{ij} = 2\min\!\parenths{E_i^2,\!E_j^2}\!\parenths{1 - \cosOf{\theta_{ij}}},
\end{equation}
where $E_i$ is the energy of particle $i$. $\theta_{ij}$ is the polar angle difference between particle $i$ and $j$. Durham algorithm can only be run at exclusive mode, which means that the clustering will stop when $d_{ij}$ is above some threshold, $d_{cut}$.

Comparing to \kt algorithm, it uses energy instead of \pT in the distance metric, and it did not have a beam jet. This is because that for the \ee collider in the past, the beam induced background was not severe and collisions energy is known, \sqrtS.

\subsection{Jet algorithm for \CLIC}

Although \CLIC is a \ee collider, the significant beam-induced background adds a large amount of energy from \ggHad process. Therefore, traditional \ee jet algorithms, like Durham algorithm, is not suitable for \CLIC environment. Studies has shown that jet algorithms for \pp collider have better performance \cite{Linssen:2012hp,LCD-Note-2010-006}.

A more recent attempt at marrying merits from both Durham and \kt algorithms has resulted in Valencia jet algorithm \cite{Boronat:2014hva}. It had shown promising improvement comparing to \kt algorithm.

\begin{comment}
Why extra C++ implementation
speed reduce O(n^3) to NlgN
y, phi space, 2D KNN problem
\end{comment}

\subsection{\y{} parameter}
\y{} parameter is a commonly used quantity to describe the transition of exclusive jet algorithm going from $N$ clustered jets to $N\!+\!1$ clustered jets. For example, $\y{23}$ would be the $d_{cut}$ value for a exclusive jet algorithm, above which the jet algorithm returns 2 jets, below which the jet algorithm returns 3 jets. 

Numerically \y{} parameter is often much smaller than one. A typically way to convert the small number to a human acceptable range is to take the minus logarithm of the number. 

\section{Flavour tagging}

The latest software package for jet flavour tagging is \lcfiplus \cite{Suehara:2015ura}. It is based on the LCFIVertex package, which was used in the simulation studies for \ILCloi \cite{Abe:2010aa,Aihara:2009ad} and \CLICcdr \cite{Linssen:2012hp}. Current software is built in mind of a future \ee collider. Although the software is modular, it will be described in order that it will be used in a physics analysis, 

The vertex finding algorithms perform vertex fitting and identify primary and secondary vertex. There is a ``V0'' particle rejection, which is when neutral particles decay or convert into a pair of charged tracks. The topology is similar to the decay of \Pbottom or \Pcharm hadrons. Hence it is important to remove the V0 particles to improve the heavy quark falvour tagging.



\subsection{MVA}
\label{sec:theoryMVA}
BDT
?likelihood
Range, discrete
Typical example:
m in a good range
log transformation

\subsection{Event shape}
Thrust
Sphericity
Aplanarity
\subsection{Jargons}
Signal
Selection
Efficiency
Significance

\qlight light quark
light lepton

Thanks computing resources. i.e. ILC VO, CLIC grid, etc. 