\chapter{Analysis technique}
\label{chap:Reconstruction}

\chapterquote{In preparing for battle I have always found that plans are useless, but planning is indispensable.}%
{Dwight D. Eisenhower}%:

Automated analysis is the only way to deal with the vast amount of data generated in the high energy physics. In the last chapter we described the automated reconstruction tools in details. This chapter is dedicated to the common automated analysis tools and techniques, which will be used in the analysis described in subsequent chapters.

For the linear collider, thanks to the high granular calorimeter, the starting point for analysis would be individual Particle Flow Objects, as well as individual tracks. Each of the PFOs encodes four-momentum and position information. For tracks, they would have momentum and position information.

However, sometimes it is interesting to group PFOs and tracks into jets, where a jet is the result of hadronisation process from high energy particles like quarks or gulons.

\section{Jet algorithm}

A jet is typically a visually obvious structure in a event display. The momentum and the direction of a jet tend to resemble the originated particle. Despite the relative easiness of identifying jets visually, it presents a challenge for a pattern recognition program to identify jets effectively and efficiently.

Early work on jet finding started in 1977 \cite{PhysRevLett.39.1436}, where later development can be found in reviews \cite{Moretti:1998qx,Salam:2009jx,Ali:2010tw}.

There are two large families of jet finding algorithm, cone based algorithm, and sequential combination algorithm. Cone based algorithm is briefly discussed in \Section{sec:pandoraConeClustering}.

Sequential combination algorithm typically calculate a pair-wise distance metric. Pairs with the smallest metric will be combined. The metric will be calculated and updated, and a pair with smallest metric will be combined. This procedure will be repeated until some stopping criterion are satisfied.

The chosen jet algorithm implementation is FastJet C++ software package \cite{Cacciari:2011ma,Cacciari:2005hq}, providing a wide range of jet finding algorithms. The implementation in Marlin software package is called MarlinFastJet. The symbols in the subsequent discussion about specific jet algorithms will follow \cite{Cacciari:2011ma}

\subsection{\kt algorithm}

One of the common sequential combination algorithms for \pp collider experiment, is longitudinally-invariant \kt algorithm \cite{Catani:1993hr,Ellis:1993tq}. In the inclusive variant, The symmetrical pair-wise distance metric between particle $i$ and $j$, and the beam distance, are defined as
\begin{equation}
&d_{ij} = d_{ji} = \min\!\parenths{\pT_{i}^{2},\!\pT_{j}^{2}}\frac{\DeltaOf{R_{ij}^{2}}}{R^{2}}, \\
&d_{iB} = \pT_{i}^2,
\end{equation}
where $\pT_{i}$ is the transverse momentum of particle $i$ with respect to the beam ($z$) direction, and $\DeltaOf{R_{ij}^{2}}$ is the measurement of angular separation of particle $i$ and $j$. Formally $\DeltaOf{R_{ij}^{2}} = \parenths{y_i - y_j}^2 + \parenths{\phi_i - \phi_j}^2$, where $y_i = \frac{1}{2}\ln\!\frac{E_i + {p_z}_i}{E_i - {p_z}_i}$ and $\phi_i$ are particle $i$'s rapidity and azimuthal angle. R is a free parameter controlling the jet radius.

If $d_{ij} < d_{iB}$, particle $i$ and $j$ are merged, with the \fourMomentum of particle $i$ updated as the sum. Otherwise, particle $i$ is set to be a final jet, and delete from the particle list. The above procedure is repeated until no particle left.

The exclusive variant is similar. First difference is that when  $d_{iB} < d_{ij}$, the particle $i$ is discarded and part of the beam jet. The second difference is that when both $d_{ij}$ and $d_{iB}$ are above some threshold, $d_{cut}$, the clustering will stop. In practise, exclusive mode allows a specified number of jets to be found, which will automatically choose the $d_{cut}$. The inclusive mode would fine as many jets as the algorithm allows.

\subsection{Durham algorithm}

Durham algorithm \cite{Catani:1991hj}, also known as \ee \kt algorithm, is commonly used \ee collider experiment. It has a single distance metric:
\begin{equation}
d_{ij} = 2\min\!\parenths{E_i^2,\!E_j^2}\!\parenths{1 - \cosOf{\theta_{ij}}},
\end{equation}
where $E_i$ is the energy of particle $i$. $\theta_{ij}$ is the polar angle difference between particle $i$ and $j$. Durham algorithm can only be run at exclusive mode, which means that the clustering will stop when $d_{ij}$ is above some threshold, $d_{cut}$.

Comparing to \kt algorithm, it uses energy instead of \pT in the distance metric, and it did not have a beam jet. This is because that for the \ee collider in the past, the beam induced background was not severe and collisions energy is known, \sqrtS.

\subsection{Jet algorithm for \CLIC}

Although \CLIC is a \ee collider, the significant beam-induced background adds a large amount of energy from \ggHad process. Therefore, traditional \ee jet algorithms, like Durham algorithm, is not suitable for \CLIC environment. Studies has shown that jet algorithms for \pp collider have better performance \cite{Linssen:2012hp,LCD-Note-2010-006}.

A more recent attempt at marrying merits from both Durham and \kt algorithms has resulted in Valencia jet algorithm \cite{Boronat:2014hva}. It had shown promising improvement comparing to \kt algorithm.

\begin{comment}
Why extra C++ implementation
speed reduce O(n^3) to NlgN
y, phi space, 2D KNN problem
\end{comment}

\subsection{\y{} parameter}
\y{} parameter is a commonly used quantity to describe the transition of exclusive jet algorithm going from $N$ clustered jets to $N\!+\!1$ clustered jets. For example, $\y{23}$ would be the $d_{cut}$ value for a exclusive jet algorithm, above which the jet algorithm returns 2 jets, below which the jet algorithm returns 3 jets.

Numerically \y{} parameter is often much smaller than one. A typically way to convert the small number to a human acceptable range is to take the minus logarithm of the number.

\section{Flavour tagging}
\label{sec:theoryFalvourTagging}

The latest software package for jet flavour tagging is \lcfiplus \cite{Suehara:2015ura}. It is based on the LCFIVertex package, which was used in the simulation studies for \ILCloi \cite{Abe:2010aa,Aihara:2009ad} and \CLICcdr \cite{Linssen:2012hp}. Current software is built in mind of a future \ee collider. Although the software is modular, it will be described in order that it will be used in a physics analysis,

The vertex finding algorithms perform vertex fitting and identify primary and secondary vertex. There is a ``V0'' particle rejection, which is when neutral particles decay or convert into a pair of charged tracks. The topology is similar to the decay of \Pbottom or \Pcharm hadrons. Hence it is important to remove the V0 particles to improve the heavy quark falvour tagging.

Jet clustering ensures that the secondary vertices and the muons identified from semileptonic decay are combined. Therefore, it is consistent with the hadronic decay. Jet algorithms used are Durham and Durham modified algorithms. 

Vertices are refined to improve the \Pbottom jet identification from c jet. Two vertices is strongly correlated to a \Pbottom jet. Hence the vertices refining will reconstruct as many secondary vertices correctly as possible.

The final flavour tagging of the jet is done using multivariate analysis, which will be discussed in \Section{sec:theoryMVA}. Using TMVA software package \cite{Hocker:2007ht}, Boosted Decision Tree classier is used. A series of flavour sensitive variables are calculated, and the classification is divided four sub-set: jet with zero, one, or two properly reconstructed vertices, or a single-track pseudovertex. For each sub-set, a jet can either be a \Pbottom jet, a \Pcharm jet, or a light flavour quark jet (\Pup, \Pdown or \Pstrange). The multiclass classifier's response is normalised across different sub-set, and they will be referred in the subsequent physics analysis as the tag value.

\section{Multivariate analysis}
\label{sec:theoryMVA}

Multivariate analysis (MVA) has become increasingly common in high energy physics. MVA can be viewed as an advanced tool for regression or classification. Comparing to the traditional cut based method, modern machine learning technique offers much improvement in data analysis. 

Software package for MVA used throughout this document is TMVA \cite{Hocker:2007ht}. 

A typical machine learning MVA classification involves two classes, also known as signal and background. A machine learning model, called classifier in TMVA, needs to be trained with training data. The model requires a set of discriminative variables, which should separate signal from background. The trained model will be applied onto the testing data, for signal extraction. Response of the model could be signal/background, or be  a number in a continuous spectrum, where the user decides the value to separate signal from background.

Strictly, there should be three statistically independent samples for the MVA. One sample is for the training. Another sample for the validation, including optimisation and checking for overfitting. The last sample is for testing. However, due to technical reason, sometimes the same sample is used for the validation and the testing. 

This classification scheme can be easily extended to multiple classes, implemented in TMVA with multiclass class. 

\subsection{Choice of models}

The model, known as the classifier in TMVA, can be as simple as cut based, likelihood or linear regression. It can be complicated as non linear tree, non linear neutral network or support vector machine. Regardless of model complexity, the choice of most optimal classifier is often data driven. Also, given the free parameters in each model, the comparison between different models without individual tuning is not rigorous. Nevertheless, as researchers in the machine learning suggested, the boosted decision tree is probably the best out-of-the-box machine learning method. Neutral network could potentially be better than the boosted decision, but it requires more tuning, and it is less intuitive to interpret the model. For these reasons, boost decision tree (BDT) is often the choice of machine learning model in the high energy physics. And it is used in various physics analysis in this document.

Before describing BDT in detail, we will first visit the traditional rectangular cut model, and the Projective Likelihood method, which is used in the photon ID in the \pandora.

\subsection{Rectangular Cut}

Probably the most intuitive model, the rectangular cut method optimise cuts to maximise some specific metric. The metric could be the signal efficiency for a particular background efficiency. Alternatively, the metric can be the significance, $\frac{S}{\rootOf{S\!+\!B}}$, where $S$ and $B$ are signal and background numbers, respectively. 

Discriminative variables gives better separation power when they are gaussian-like and statistically independent. Therefore it is common to decoorelate  the variables and gaussian transform them before using the rectangular cut MVA.

Because its simplicity, the cut method is often performed manual, much more often in the time pre-date the wide spread of machine learning methods. It is still commonly used for the pre-selection step before the MVA, and other simple usages. Unless specified, the optimal cuts proposed in this document for various physics analysis are found using the rectangular cut method manually. 

\subsection{Projective Likelihood}

Projective likelihood model (PDE) is used in \pandora for the photon ID due to its simplicity and low requirement on computing resources. 

PDE implemented in the TMVA calculates the probability density for each discriminative variable, for signal and background. The overall signal and background likelihood are defined as products of the individual probability density. The likelihood ratio, $R$, is then defined as the signal likelihood over signal plus background likelihood.

TMVA implementation also fits an underlying function to the probability density. The \pandora implementation simply uses binned likelihood ratio, $R$, as the output, due to the simplicity. The sub-categories for the \pandora implementation are determined by the cluster energy.

Similarly to the rectangular cut method, PDE works better with decorrelated, gaussian like variables. The \pandora implementation did not decorrelate nor transform the variables, to keep implementation fast. 


\subsection{Boosted decision tree}



\subsection{Optimisation and overfitting}







BDT
?3likelihood
Range, discrete
Typical example:
m in a good range
log transformation

\section{Event shape}
Thrust
Sphericity
Aplanarity
\subsection{Jargons}
Signal
Selection
Efficiency
Significance

\qlight light quark
light lepton

Thanks computing resources. i.e. ILC VO, CLIC grid, etc. 