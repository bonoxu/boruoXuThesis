%% The "\appendix" call has already been made in the declaration
%% of the "appendices" environment (see thesis.tex).
\chapter{Generation parameters}
\label{app:photon}

\chapterquote{Opportunities multiply as they are seized.}
{Du Mu, 803 $-$ 852}

%Appendixes (or should that be ``appendices''?) make you look really clever, 'cos
%it's like you had more clever stuff to say than could be fitted into the main
%bit of your thesis. Yeah. So everyone should have at least three of them\dots

Particle masses and widths used to generate SM samples for studies with \CLIC detectors, used in \Chapter{chap:DoubleHiggs}, are listed in \Table{tab:pandoraCLICparticleMass}. The Higgs boson mass is specified for individual samples.

\begin{table}[htbp]
\centering
\smallskip
\begin{tabular}{l r  r }
\hline
\hline
Particle &  Mass (\uprightMath{GeV/c^2}) & Width (\uprightMath{GeV/c^2}) \\
\hline
\Pup, \Pdown, \Pstrange quarks& 0 &  0\\
\Pcharm quark& 0.54 &  0\\
\Pbottom quark& 2.9 &  0\\
\Ptop quark& 174 & 1.37\\
\PW boson & 80.45 &  2.071\\
\PZ boson & 91.188 &  2.478\\
\hline
\hline
\end{tabular}
\caption[Masses of quarks and bosons used for  generating Standard Model samples.]%
{Particle masses and widths used for the generation of SM samples for \CLIC detectors, taken from \cite{Linssen:2012hp}.}
\label{tab:pandoraCLICparticleMass}
\end{table}

