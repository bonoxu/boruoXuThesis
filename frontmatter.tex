%% Title
\titlepage[of King's College]{%
  A dissertation submitted to the University of Cambridge\\ for the degree of Doctor of Philosophy}

%% Abstract
\begin{abstract}%[\smaller \thetitle\\ \vspace*{1cm} \smaller {\theauthor}]
  %\thispagestyle{empty}
An electron-positron linear collider is an option for future large particle accelerator projects. Such a collider would focus on precision tests of the higgs boson properties. This thesis describes
several studies related to the optimisation of high granular calorimeters. Three main areas were covered.

The performance of photon reconstruction is improved. Photon reconstruction algorithms were developed within PandoraPFA, a world-leading pattern-recognition software for particle flow calorimetry. A sophisticated pattern recognition algorithm was implemented, which uses the topological properties of electromagnetic showers to identify photon candidates and separate them from nearby particles. It performs clustering of the energy deposits in the detector, followed by topological characterisation of the clusters, with the results being considered by a multivariate likelihood analysis. This algorithm leads to a significant improvement in the reconstruction of both single photons and multiple photons in high energy jets.

Reconstruction and classification of tau lepton decay modes were studied. Tau decay products, such as photons, were reconstructed as separate entities. Utilising high granular calorimeters, the resolution of energy and invariant mass of the tau decay products is improved. A hypothesis test was performed for expected decay final states. A multivariate analysis was trained to classify decay final states with a data-driven machine learning method. The performance of tau decay classification is used for the electromagnetic calorimeter optimisation at the ILC or CLIC.

Sensitivity of higgs couplings at the CLIC was studied, using simulated double Higgs boson production. Algorithms were developed to identify isolated high energy leptons, and results were fed into a multivariate analysis. The study was done for two CLIC energy scenarios. This sensitivity study of triple and quartic Higgs self-couplings is a part of scientific cases for the CLIC. This work provides further motivation for high granular particle flow calorimetry for a future electron-positron linear collider.
%  \LHCb is a \bphysics detector experiment which will take data at
%  the \unit{14}{\TeV} \LHC accelerator at \CERN from 2007 onward\dots
\end{abstract}


%% Declaration
\begin{declaration}
  This dissertation is the result of my own work, except where explicit
  reference is made to the work of others, and has not been submitted
  for another qualification to this or any other university. This
  dissertation does not exceed the word limit for the respective Degree
  Committee.
  \vspace*{1cm}
  \begin{flushright}
    Boruo Xu
  \end{flushright}
\end{declaration}


%% Acknowledgements
\begin{acknowledgements}
  %Of the many people who deserve thanks, some are particularly prominent,
  %such as my supervisor\dots
There are many people that I would to thank  for their help in my pursuit of a PhD degree. First of all, I would like express my most sincere gratitude to my parents, for their finical support and moral support. They have been supporting me for all this many years. Especially, when the PhD study became an intense and stressful exercise, they were able to put up with me and not abandon me. During a few months when I was really worried about not able to finish the PhD program and facing unemployment, they talked me through and gave me much consoling  when I needed.

The next person I would like to thank is my supervisor, Mark Thomson. I was lucky to follow him to embark an incredible journey on an exciting project. I have received much useful guidance from him on numerous occasions. On one occasion, which influenced me greatly, was in the very early stage of my PhD study. I managed to make improvements to some algorithms. However, a study suggested that my improved algorithms were not as good as a rival algorithm by a certain metric. Feeling defeated and eager to prove myself, I wanted to repeat the studies just to prove that my algorithms are better. Mark suggested that it is more important to have a project to understand physics, rather than competing for the best performance defined by some arbitrary metrics, which taught me the importance of having the right priority in work, rather than engaging in meaningless competition, however tempting it may be.

I would also like to thank John Marshall for his constant support over the last four years. A large part of the improvement in coding skills is because of the help from John. There was a couple of months, where I had written my working algorithms in ugly codes, and had to rewrite my codes to meet \pandora code standard. This refactorisation exercise indeed taught me a lot about the C++ coding concepts, as well as good coding habits. It was also him who introduced me to the wonderful world of git, which I hated in the beginning. Nevertheless, I was fortunate to have John as my second supervisor and coding mentor.

I was also extremely fortunate to have Steven Green as my colleague and my cherished friend. Other than the lovely, occasionally frustrating, four years that we spent in the same office,   I was privileged to spend two years with Steve sampling the fine ale from local pubs on a regular basis. After the infamous ``gin'' incident, which was a great night, we continued to share our love of ale and pork scratchings in a much more civilised fashion. I was also honoured to be the usher on Steve's wedding. The wedding was great. And we should have more boardgame nights.

Before moving on to external collaborators, I would also like to thank Joris de Vries for providing entertainments in the office, for embarking on numerous pub trips together, and for suffering together in the ``ceiling'' incident. I would also like to thank Jack Anthony and Andy Smith for enduring me in the same office, and the rest of the Cambridge HEP group for their support.

I would like to thank Philipp Roloff for his teaching on various techniques in a physics analysis; Rosa Simoniello for collaborating on the double Higgs production analysis. The analysis would take much longer to finish without their help. I would also like to thank Andr\'{e} Sailer and Marko Petric for their support with the \CLIC grid computing system. At this time of this thesis is written, I should probably still be the top user on the grid system, in terms of the cpu time, much thanks to their help. I also have to thank Andr\'{e} for introducing me to Caf\'{e} de l'aviation. It was the best steak that I had in Europe. My gratitude also goes to Lucie Linssen, who was very kind to fund several of my trips to CERN. It was an enjoyable experience to work in CERN and it would be impossible without Lucie's support. I would also like to thank the rest of CLICdp group in CERN for the friendly and the useful collaboration during my PhD.

My friends in Cambridge, whom I probably see on daily basis, deserve my a lot of my appreciation. It is them who made my PhD study in Cambridge lively and fun. I am again very luck not only to gain a PhD degree after another four years in Cambridge, but also to gain a group of good friends.

Apart from all the people that I have thanked above, there are a few extra people who proof-read my thesis: David Arvidsson, Sophie Morrison, and Laure-Anne Vincent. Thank you for the constructive suggestionss on my thesis.

Because of all the people that I have thanked, and those who I forget to thank, I was privileged to be able to spend four years to research on a topic that is truly interesting.


\end{acknowledgements}


%% Preface
%\begin{preface}
%This will be my preface. Where is Wolly?
%  This thesis describes my research on various aspects of the \LHCb
%  particle physics program, centred around the \LHCb detector and \LHC
%  accelerator at \CERN in Geneva.

 % \noindent
 % For this example, I'll just mention \ChapterRef{chap:SomeStuff}
 % and \ChapterRef{chap:MoreStuff}.
%\end{preface}

%% ToC
\tableofcontents


%% Strictly optional!
\frontquote{You cannot open a book without learning something.}%
{Confucius, 551 BC $-$ 479 BC}

%
%Victorious warriors win first and then go to war, while defeated warriors go to war first and then seek to win.

%{Sun Tzu, 544 BC $-$ 496 BC}
%\frontquote{A Higgs-Boson walks into a church, \\
%the priest says \\
%``We don't allow Higgs-Bosons in here.''. \\
%The Higgs-Boson says \\
%``But without me, how can you have mass?''.}
%  {Reddit}
%% I don't want a page number on the following blank page either.

%  Writing in English is the most ingenious torture\\
%  ever devised for sins committed in previous lives.}%
%  {James Joyce}

%{\begin{CJK*}{UTF8}{zhsong}
%三人行,必有我師焉。
%\end{CJK*}}\\
\thispagestyle{empty}
