\chapter{Summary and outlook}
\label{chap:summary}

%% Restart the numbering to make sure that this is definitely page #1!
%\pagenumbering{arabic}

%% Note that the citations in this chapter use the journal and
%% arXiv keys: I used the SLAC-SPIRES online BibTeX retriever
%% to build my bibliography. There are also quite a few non-standard
%% macros, which come from my personal collection. You can have them
%% if you want, or I might get round to properly releasing them at
%% some point myself.

\chapterquote{If you know the enemy and know yourself, you need not fear the result of a hundred battles.}%
{Sun Tzu, 544 BC-496 BC}


%Introduction

%\section{Summary}

This chapter summarises key results presented in the analysis in previous chapters.

In \Chapter{chap:Photon}, a set of photon reconstruction related algorithms are presented. The photon fragments produced during the event reconstruction have been greatly reduced. The photon separation power has improved and the jet reconstruction has been improved, as a result of a better photon reconstruction.

In \Chapter{chap:Tau}, a high classification rate of the tau lepton decay mode is achieved. The classification is applied to different electromagnetic calorimeter cell sizes with different centre-of-mass energies. Wit the classification, a proof-of-principle analysis shows the tau polarisation correlations with \ZToTauTau decay where \tauToPion can be observed with \ILD detector model. With a similar study of \HiggsToTauTau,  the tau polarisation correlations can be used to separate \PHiggs from \PZ.
 
 
 In \Chapter{chap:DoubleHiggs}, the analyses of the \eeToHHbbWWFull channel for the Compact Linear Collider at \rootS{1.4} and \rootS{3} are performed. The significance of the signal events are 0.56 and 1.09 assuming an integrated luminosity of 1500$fb^{-1}$ and 2000$fb^{-1}$, for \rootS{1.4}  and \rootS{3} respectively. The extraction of the Higgs trilinear self coupling, \gHHH, and the quartic coupling, \gWWHH, is provided.

