\chapter{Summary}
\label{chap:summary}

%% Restart the numbering to make sure that this is definitely page #1!
%\pagenumbering{arabic}

%% Note that the citations in this chapter use the journal and
%% arXiv keys: I used the SLAC-SPIRES online BibTeX retriever
%% to build my bibliography. There are also quite a few non-standard
%% macros, which come from my personal collection. You can have them
%% if you want, or I might get round to properly releasing them at
%% some point myself.

\chapterquote{To know what you know and what you do not know, that is true knowledge.}%
{Confucius, 551 BC $-$ 479 BC}


%Sun Tzu, 544 BC $-$ 496 BC

%Introduction

%\section{Summary}

This chapter summarises key results presented in analyses in this thesis.

%Using the \ILD detector model, the single photon reconstruction efficiency is above 98\% for photons with energies above 2\GeV, and above 99.5\% for photons with energies above 100\,GeV. For the photon fragment reduction performance, using a two-photon-per-event sample with photon energies of  500\,GeV and 50\,GeV, the average numbers of photons and particles beyond a distance separation of 20\,mm  are both less than 2.05, where the true value is 2. Photon pairs with the same energy, for example, 500\,GeV$-$500\,GeV photon pair and 10\,GeV$-$10\,GeV photon pair, start to be resolved at a distance separation of 6\,mm, which is about one \ECAL cell length. Photon pairs with different energies, for example 500\,GeV$-$50\,GeV and  100\,GeV$-$10\,GeV pairs, start to be resolved at a distance separation of 10\,mm, which is about two \ECAL cells length. At a distance separation of 20\,mm, 500\,GeV$-$500\,GeV photon pairs are fully resolved, whereas approximately only 60\% of 10\,GeV$-$10\,GeV photon pairs are resolved.

 In \Chapter{chap:Photon}, a set of photon reconstruction algorithms developed in \pandora are discussed. Using the \ILD detector model, the single photon reconstruction efficiency is above 98\% for photons with energies above 2\GeV, and above 99.5\% for photons with energies above 100\,GeV. The photon fragments produced during the event reconstruction have been greatly reduced. The ability to separate spatially close photons and the jet energy resolution have improved, as a result of a better photon reconstruction. Using a two-photon sample with photon energies of  500\,GeV and 50\,GeV, the average numbers of photons and particles beyond a distance separation of 20\,mm  are both less than 2.05, where the true value is 2. The distance of the minimal resolved photon pairs is reduced to 6\,mm for two photons with the same energy, and 10\,mm for two photons with different energies. The jet energy resolution has been improved for \rootSGeV{360} and 500\,GeV. The photon confusion terms, except at \rootSGeV{91}, have been reduced to 0.9\%.


%500 - 500\,GeV photon pair and 10 - 10\,GeV photon pair start to be resolved at 6\,mm apart, which is about 1 \ECAL cell. For photon pairs with different energies,  for example 500 - 50\,GeV pair and  100 - 10\,GeV pair, start to be resolved at 10\,mm apart, which is about 2 \ECAL cells. At 20\,mm apart, two photons in  500 - 500\,GeV pair are fully resolved, where approximately 60\% of two photons in 10 - 10\,GeV pair are resolved.

In \Chapter{chap:Tau}, a high classification rate of the tau lepton seven major decay modes is achieved using the \eeTauTau events in the \ILD detector. The tau decay mode classification is used for the \ECAL optimisation study with different \ECAL cell sizes and different centre-of-mass energies. The efficiency of the tau decay mode classification  decreases with an increase of the centre-of-mass energy and with the  increasing \ECAL cell sizes. The sensitivity of \tauHad to different cell sizes is stronger at high centre-of-mass energies.  For the \ILC at \rootSGeV{250} or \CLIC at \rootSGeV{350}, an \ECAL size of 10\,mm or fewer is sufficient to achieve a \tauHad of 92\%. For a linear collider operating at a centre-of-mass energy above a few hundred GeVs, such as the \ILC at \rootSGeV{500} or \CLIC at \rootS{1.4} or 3\,TeV, it is preferable to have a small \ECAL cell size, i.e. 3\,mm,  for the best tau decay mode separation, as \tauHad decreases drastically with an increasing \ECAL cell size.

\CHAPTER{chap:Tau2Mini} presents a proof-of-principle demonstration that the generator-level pion energy fraction correlation can be reconstructed at the analysis level. The analysis contains several important steps: tau identification; kinematic reconstruction of the energies of the taus; the classification of the  \tauToPionBoth decay mode; and the reconstruction of the tau pair polarisation correlations.


% At a centre-of-mass energy of 100\,GeV, the tau hadronic decay correct classification efficiency, \tauHad, decreases from 94\% at 3\,mm \ECAL cell size, to 91\% at 20\,mm cell size. Most significant decrease in the \tauHad occurs at  a centre-of-mass energy of 500\,GeV, where the \tauHad decreases from 92\% at 3\,mm cell size, to 78\% at 20\,mm cell size. The increase in \ECAL cell sizes has a larger impact on the performance of the tau decay classification for centre-of-mass energies above 200\,GeV, and it is more beneficial to have a small \ECAL cell size at these high centre-of-mass energies.

 In \Chapter{chap:DoubleHiggs}, the analyses of the \eeToHH $\to$ \HepProcess{ \Pbottom \APbottom \PWplus \PWminus \Pnue \APnue} channel for \CLIC at \rootS{1.4} and \rootS{3} are performed. The significance of the signal events are 0.56 and 1.09,  assuming an integrated luminosity of 1500\,\uprightMath{fb^{-1}} and 2000\,\uprightMath{fb^{-1}}, for \rootS{1.4}  and \rootS{3} respectively.  The uncertainty on measurement of the Higgs trilinear self coupling, \gHHH, from  \eeToHH $\to$ \HepProcess{ \Pbottom \APbottom \PWplus \PWminus \Pnue \APnue} analysis is obtained:
\begin{equation}
\frac{\Delta\gHHH}{\gHHH}=
\begin{cases}
  218\%, & \mbox{at \rootS{1.4}, }  \\
  135\%, & \mbox{at \rootS{3}}.
\end{cases}
\end{equation}
When analysis at both \eeToHHbbWW and \eeToHHbbbb sub-channels are combined at \rootS{3}, assuming an integrated luminosity of 3000\,\uprightMath{fb^{-1}}, the simultaneous extraction of the uncertainty on the measurement of the \gHHH and \gWWHH yields:
\begin{equation}
\frac{\Delta\gWWHH}{\gWWHH}  \simeq 4.9\% , & \text{ for \gHHH = $\gHHHSM$},\\
\frac{\Delta\gHHH}{\gHHH}  \simeq 29\% , & \text{ for \gWWHH = $\gWWHHSM$}.
\end{equation}
The statistical precisions on the measurements of \gWWHH and \gHHH are much better at \CLIC than at current \LHC or  high-luminosity upgraded \LHC \cite{Contino:2010mh}.
