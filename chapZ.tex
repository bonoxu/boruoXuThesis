\chapter{Summary}
\label{chap:summary}

%% Restart the numbering to make sure that this is definitely page #1!
%\pagenumbering{arabic}

%% Note that the citations in this chapter use the journal and
%% arXiv keys: I used the SLAC-SPIRES online BibTeX retriever
%% to build my bibliography. There are also quite a few non-standard
%% macros, which come from my personal collection. You can have them
%% if you want, or I might get round to properly releasing them at
%% some point myself.

\chapterquote{If you know the enemy and know yourself, you need not fear the result of a hundred battles.}%
{Sun Tzu, 544 BC - 496 BC}


%Introduction

%\section{Summary}

This chapter summarises key results presented in analyses in previous chapters. In \Chapter{chap:Photon}, a set of photon reconstruction algorithms developed in \pandora are presented. The photon fragments produced during the event reconstruction have been greatly reduced. The photon separation power and the jet energy resolution have improved, as a result of a better photon reconstruction.

For the single photon reconstruction, the efficiency is above 98\% for photons with energies above 2\GeV, and above 99.5\% for photons with energies above 100\,GeV. To quantise the photon fragment reduction, for a  500 - 50\,GeV photons pair sample, the average number of photons and particles beyond 20\,mm apart are both less than 2.05, where the true value is 2. For the photon separation power, 500 - 500\,GeV photon pair and 10 - 10\,GeV photon pair start to be resolved at 6\,mm apart, which is about 1 \ECAL cell. For photon pairs with different energies,  for example 500 - 50\,GeV pair and  100 - 10\,GeV pair, start to be resolved at 10\,mm apart, which is about 2 \ECAL cells. At 20\,mm apart, two photons in  500 - 500\,GeV pair are fully resolved, where approximately 60\% of two photons in 10 - 10\,GeV pair are resolved.

In \Chapter{chap:Tau}, a high classification rate of the tau lepton seven major decay modes is achieved. The classification is applied to different electromagnetic calorimeter cell sizes with different centre-of-mass energies. The tau hadronic decay correct classification efficiency, \tauHad,  is used as the performance metric. At \rootSGeV{100}, the \tauHad decreases from 94\% at 3\,mm cell size, to 91\% at 20\,mm cell size. Most significant decrease in the \tauHad occurs at  \rootSGeV{500}, where the \tauHad decreases from 92\% at 3\,mm cell size, to 78\% at 20\,mm cell size.
The increase in \ECAL cell sizes has a larger impact in tau decay classification at high centre-of-mass energies. With decay products spatially close at high centre-of-mass energies, it is more beneficial to have a smaller \ECAL cell size to reconstruct individual particle.

With the developed tau decay mode classification, a proof-of-principle analysis shows the tau pair polarisation correlations with \ZToTauTau decay where \tauToPion can be observed with \ILD detector model. A good match of the tau pair polarisation correlations between the reconstruction and the Monte Carlo simulation is also achieved.   With a similar study of \HiggsToTauTau,  the tau polarisation correlations can be used to as a signature to identify  Higgs boson from \PZ boson.




 In \Chapter{chap:DoubleHiggs}, the analyses of the \eeToHH $\to$ \HepProcess{ \Pbottom \APbottom \PWplus \PWminus \Pnue \APnue} channel for the Compact Linear Collider at \rootS{1.4} and \rootS{3} are performed. The significance of the signal events are 0.56 and 1.09,  assuming an integrated luminosity of 1500$fb^{-1}$ and 2000$fb^{-1}$, for \rootS{1.4}  and \rootS{3} respectively.  The uncertainty on measurement of the Higgs trilinear self coupling, \gHHH, from  \eeToHH $\to$ \HepProcess{ \Pbottom \APbottom \PWplus \PWminus \Pnue \APnue} analysis is obtained:
\begin{equation}
\frac{\Delta\gHHH}{\gHHH}=
\begin{cases}
  218\%, & \mbox{at \rootS{1.4}, }  \\
  135\%, & \mbox{at \rootS{3}}.
\end{cases}
\end{equation}

When analysis at both \eeToHHbbWW and \eeToHHbbbb sub-channels are combined at \rootS{3} to improve the measurement, the simultaneous extraction of the uncertainty on the measurement of the \gHHH and \gWWHH yeilds:
\begin{equation}
\frac{\Delta\gWWHH}{\gWWHH} \simeq 4.9\% \text{ for \gHHH = $\gHHH_{,SM}$}
\end{equation}
\begin{equation}
\frac{\Delta\gHHH}{\gHHH} \simeq 29\% \text{ for \gWWHH = $\gWWHH_{,SM}$}
\end{equation}
