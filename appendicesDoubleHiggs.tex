%% The "\appendix" call has already been made in the declaration
%% of the "appendices" environment (see thesis.tex).
\chapter{Double Higgs Boson Production Analysis}
\label{app:doubleHiggs}

\chapterquote{I was an adventurer like you, then I took an arrow in the knee.}
{The town guard, Skyrim, 2011}

%Appendixes (or should that be ``appendices''?) make you look really clever, 'cos
%it's like you had more clever stuff to say than could be fitted into the main
%bit of your thesis. Yeah. So everyone should have at least three of them\dots

Here are extra tables and plots for the \Chapter{chap:DoubleHiggs}.

\section{Hadronic decay at \rootS{3} analysis}

\begin{table}[!tbp]\centering
% TODO fix lumi correction for e gamma, gamma e
% TODO change some of sample cross section for  electron-photon interaction with four quarks and a neutrino final state
%\small{
\begin{tabular}{lrrrr}
\hline \hline
 \multicolumn{1}{L{3.5cm}}{\rootS{3}} &  \multicolumn{1}{R{2cm}}{Expected number of events}  & \multicolumn{1}{R{1.5cm}}{Lepton veto} & \multicolumn{1}{R{1.5cm}}{Mutually exclusive} & \multicolumn{1}{R{1.5cm}}{Jet pairing} \\
\hline
\eeToHH $\to$ \\
\HepProcess{ \Pbottom \APbottom \PWplus \PWminus \Pnue \APnue}, hadronic             &146.0& 80.9\% & 72.8\% & 72.1\%\\
\hline
\eeToHH $\to$ \\
\HepProcess{ \Pbottom \APbottom \Pbottom \APbottom \Pnue \APnue}             &355.0& 83.5\% & 20.5\% & 20.5\% \\
\eeToHH $\to$ other & 675.0 & 40.1\% & 34.3\% & 20.5\% \\
\hline
\eeTo{\qlight \qlight \PHiggs \Pnu \APnu}  & 6120 & 67.7\% & 61.9\% & 61.9\%\\
\eeTo{\Pcharm \APcharm \PHiggs \Pnu \APnu}  & 2300 & 69.1\%& 53.0\%& 48.8\%\\
\eeTo{\Pbottom \APbottom \PHiggs \Pnu \APnu}  & 3560 & 70.1\%& 30.9\%& 30.6\%\\

\eeTo{ \Pquark \Pquark \Pquark \Pquark}   &   1093000& 62.4\% & 44.9\%&34.9\%\\
\eeTo{ \Pquark \Pquark \Pquark \Pquark \Plepton \Plepton}& 338600& 21.4\%& 19.6\%& 13.3\%\\
\eeTo{ \Pquark \Pquark \Pquark \Pquark \Plepton \Pnu}& 213200 & 23.3\%& 19.5\%& 16.3\%\\
\eeTo{ \Pquark \Pquark \Pquark \Pquark \Pnu \APnu} & 143000& 80.7\%& 71.4\%& 50.7\%\\

\eeTo{ \Pquark \Pquark} &  5897800 & 72.9\%& 63.9\%& 55.4\%\\
\eeTo{ \Pquark \Pquark \Plepton \Pnu} &  11121800 & 34.0\%& 24.7\%& 20.5\%\\
\eeTo{ \Pquark \Pquark \Pl \Pl} &  6639200 & 43.1\%& 41.7\%& 37.0\%\\
\eeTo{ \Pquark \Pquark \Pnu \Pnu} & 2635000 &84.6\%& 63.8\%& 53.2\% \\
\hline
\egamma{\Pepm}{\Pphoton}{BS}{\Pepm \Pquark \Pquark \Pquark \Pquark} & 4007354  & 31.0\%& 28.2\%& 21.1\%\\
%\egamma{\Pem}{\Pphoton}{BS}{\Pem \Pquark \Pquark \Pquark \Pquark} & 2004388.1  & 30.8\%& 28.0\%& 21.0\%\\
%\egamma{\Pep}{\Pphoton}{BS}{\Pep \Pquark \Pquark \Pquark \Pquark} & 2002334.1 & 31.1\%& 28.3\%& 21.1\%\\
\egamma{\Pepm}{\Pphoton}{EPA}{\Pepm \Pquark \Pquark \Pquark \Pquark} & 1151200& 15.9\%& 14.5\%& 10.9\%\\
%\egamma{\Pem}{\Pphoton}{EPA}{\Pem \Pquark \Pquark \Pquark \Pquark} & 575600.0& 15.9\%& 14.5\%& 10.8\%\\
%\egamma{\Pep}{\Pphoton}{EPA}{\Pep \Pquark \Pquark \Pquark \Pquark}  & 575600.0 & 15.9\% & 14.5\%& 10.9\%\\
\egamma{\Pepm}{\Pphoton}{BS}{\Pnu \Pquark \Pquark \Pquark \Pquark}& 829184  & 78.3\%& 68.8\%& 53.3\%\\
%\egamma{\Pem}{\Pphoton}{BS}{\Pnu \Pquark \Pquark \Pquark \Pquark}& 414750.0  & 78.2\%& 68.8\%& 53.5\%\\
%\egamma{\Pep}{\Pphoton}{BS}{\APnu \Pquark \Pquark \Pquark \Pquark}& 414434.0 & 78.3\% & 68.7\%& 53.0\%\\
\egamma{\Pepm}{\Pphoton}{EPA}{\Pnu \Pquark \Pquark \Pquark \Pquark}& 216800  & 39.6\% & 35.0\%& 26.9\%\\
%\egamma{\Pem}{\Pphoton}{EPA}{\Pnu \Pquark \Pquark \Pquark \Pquark}& 108400.0  & 39.6\% & 35.0\%& 26.9\%\\
%\egamma{\Pep}{\Pphoton}{EPA}{\APnu \Pquark \Pquark \Pquark \Pquark}& 108400.0  & 39.5\%& 35.0\%& 26.8\% \\
\egamma{\Pepm}{\Pphoton}{BS}{\Pquark \Pquark \PHiggs \Pnu} & 185018.0  & 64.0\% &55.4\%& 49.8\% \\
%\egamma{\Pem}{\Pphoton}{BS}{\Pquark \Pquark \PHiggs \Pnu} & 92588.0  & 64.1\% &55.4\%& 49.8\% \\
%\egamma{\Pep}{\Pphoton}{BS}{\Pquark \Pquark \PHiggs \Pnu} & 92430.0 & 63.9\% & 55.4\% & 49.8\% \\
\egamma{\Pepm}{\Pphoton}{EPA}{\Pquark \Pquark \PHiggs \Pnu} & 46800 & 32.9\% &28.8\% & 25.9\% \\
%\egamma{\Pem}{\Pphoton}{EPA}{\Pquark \Pquark \PHiggs \Pnu} & 23400.0 & 33.2\% &29.0\% & 26.1\% \\
%\egamma{\Pep}{\Pphoton}{EPA}{\Pquark \Pquark \PHiggs \Pnu} & 23400.0   & 32.6\% & 28.6\% & 25.7\% \\
\hline
\gammagamma{\Pphoton}{BS}{\Pphoton}{BS}{ \Pquark \Pquark \Pquark \Pquark}& 18009414  & 71.6\%& 65.5\%& 49.4\%\\
\gammagamma{\Pphoton}{BS}{\Pphoton}{EPA}{ \Pquark \Pquark \Pquark \Pquark}& 3824548  &44.3\%& 40.6\%& 30.6\%\\
\gammagamma{\Pphoton}{EPA}{\Pphoton}{BS}{ \Pquark \Pquark \Pquark \Pquark}& 3828498 & 44.3\%& 40.7\%& 30.7\%\\
\gammagamma{\Pphoton}{EPA}{\Pphoton}{EPA}{ \Pquark \Pquark \Pquark \Pquark}& 805400 & 29.0\% & 26.7\% & 20.1\%\\
\hline \hline
\end{tabular}

\caption
{Number of events and fraction of events passing lepton veto, the mutually exclusive cuts, and the jet pairing  for the signal and background events at \rootS{3}, assuming an integrated luminosity of 2000\,$fb^{-1}$. The selection efficiencies are presented in a ``flow'' fashion. Every selection cut contains all the cuts to the left of it. \Pquark can be \Pup, \Pdown, \Pstrange, \Pbottom or \Ptop. Unless specified, \Pquark, \Plepton and \Pnu represent either particles or the corresponding anti-particles. \Pphoton(BS) represents a real photon from beamstrahlung (BS). \Pphoton(EPA) represents a ``quasi-real'' photon, simulated with the Equivalent Photon Approximation.
}
\label{tab:doubleHiggs3TeVPreslection}
\end{table}



\begin{table}[!tbp]\centering

\begin{tabular}{lrr}
\hline \hline
 \multicolumn{1}{L{0.3\textwidth}}{Channel} &  \multicolumn{1}{R{0.3\textwidth}}{$m_{\HH}$>150GeV}  & \multicolumn{1}{R{0.3\textwidth}}{$B_1$>0.7} \\
\hline
\eeToHH $\to$ \\
\HepProcess{ \Pbottom \APbottom \PWplus \PWminus \Pnue \APnue}, hadronic             & 71.7\% & 61.8\%\\
\hline
\eeToHH $\to$ \\
\HepProcess{ \Pbottom \APbottom \Pbottom \APbottom \Pnue \APnue}             & 20.2\% & 18.8\% \\
\eeToHH $\to$ other & 30.2\% & 20.0\% \\
\hline
\eeTo{\qlight \qlight \PHiggs \Pnu \APnu}   & 53.1\% & 36.0\%\\
\eeTo{\Pcharm \APcharm \PHiggs \Pnu \APnu} & 43.8\%& 26.3\%\\
\eeTo{\Pbottom \APbottom \PHiggs \Pnu \APnu} & 29.6\%& 25.9\%\\

\eeTo{ \Pquark \Pquark \Pquark \Pquark}   & 26.5\%& 1.7\%\\
\eeTo{ \Pquark \Pquark \Pquark \Pquark \Plepton \Plepton} & 12.8\%& 0.7\%\\
\eeTo{ \Pquark \Pquark \Pquark \Pquark \Plepton \Pnu} & 16.0\%& 7.9\%\\
\eeTo{ \Pquark \Pquark \Pquark \Pquark \Pnu \APnu} &49.7\%& 9.0\%\\

\eeTo{ \Pquark \Pquark} & 8.3\%& 1.4\%\\
\eeTo{ \Pquark \Pquark \Plepton \Pnu} & 6.0\%& 0.1\%\\
\eeTo{ \Pquark \Pquark \Pl \Pl} & 1.9\%& 0.4\%\\
\eeTo{ \Pquark \Pquark \Pnu \Pnu} & 16.6\%& 3.1\% \\
\hline
\egamma{\Pepm}{\Pphoton}{\BS}{\Pepm \Pquark \Pquark \Pquark \Pquark}& 19.4\%& 0.7\%\\
%\egamma{\Pem}{\Pphoton}{BS}{\Pem \Pquark \Pquark \Pquark \Pquark}& 19.3\%& 0.7\%\\
%\egamma{\Pep}{\Pphoton}{BS}{\Pep \Pquark \Pquark \Pquark \Pquark} & 19.4%& 0.7\%\\
\egamma{\Pepm}{\Pphoton}{\EPA}{\Pepm \Pquark \Pquark \Pquark \Pquark} & 9.9\%& 0.4\%\\
%\egamma{\Pem}{\Pphoton}{EPA}{\Pem \Pquark \Pquark \Pquark \Pquark} & 09.9\%& 0.4%\\
%\egamma{\Pep}{\Pphoton}{EPA}{\Pep \Pquark \Pquark \Pquark \Pquark}   & 9.9\%& 0.4\%\\
\egamma{\Pepm}{\Pphoton}{\BS}{\Pnu \Pquark \Pquark \Pquark \Pquark} & 51.3\%& 16.4\%\\
%\egamma{\Pem}{\Pphoton}{BS}{\Pnu \Pquark \Pquark \Pquark \Pquark} & 51.5\%& 16.8\%\\
%\egamma{\Pep}{\Pphoton}{BS}{\APnu \Pquark \Pquark \Pquark \Pquark} & 51.1\%& 15.9\%\\
\egamma{\Pepm}{\Pphoton}{\EPA}{\Pnu \Pquark \Pquark \Pquark \Pquark} & 26.0\%& 7.7\%\\
%\egamma{\Pem}{\Pphoton}{EPA}{\Pnu \Pquark \Pquark \Pquark \Pquark} & 26.0\%& 7.9\%\\
%\egamma{\Pep}{\Pphoton}{EPA}{\APnu \Pquark \Pquark \Pquark \Pquark}& 25.9\%& 7.5\% \\
\egamma{\Pepm}{\Pphoton}{\BS}{\Pquark \Pquark \PHiggs \Pnu} &47.9\%& 30.3\% \\
%\egamma{\Pem}{\Pphoton}{BS}{\Pquark \Pquark \PHiggs \Pnu} &47.9\%& 30.2\% \\
%\egamma{\Pep}{\Pphoton}{BS}{\Pquark \Pquark \PHiggs \Pnu} & 47.9\% & 30.3\% \\
\egamma{\Pem}{\Pphoton}{\EPA}{\Pquark \Pquark \PHiggs \Pnu} & 25.0\% & 15.8\% \\
%\egamma{\Pem}{\Pphoton}{EPA}{\Pquark \Pquark \PHiggs \Pnu} & 25.2\% & 16.0\% \\
%\egamma{\Pep}{\Pphoton}{EPA}{\Pquark \Pquark \PHiggs \Pnu} & 24.8\% & 15.6\% \\
\hline
\gammagamma{\Pphoton}{\BS}{\Pphoton}{\BS}{ \Pquark \Pquark \Pquark \Pquark}& 44.5\%& 1.7\%\\
\gammagamma{\Pphoton}{\BS}{\Pphoton}{\EPA}{ \Pquark \Pquark \Pquark \Pquark}& 27.4\%& 1.0\%\\
\gammagamma{\Pphoton}{\EPA}{\Pphoton}{\BS}{ \Pquark \Pquark \Pquark \Pquark}& 27.5\%& 1.0\%\\
\gammagamma{\Pphoton}{\EPA}{\Pphoton}{\EPA}{ \Pquark \Pquark \Pquark \Pquark} & 18.0\% & 0.7\%\\
\hline \hline
\end{tabular}
\caption
{List of signal and background events after each pre-selection cut at \rootS{3}.  The selection efficiencies are presented in a ``flow'' fashion. Every selection cut contains all the cuts to the left of it and all  cuts in \Table{tab:doubleHiggs3TeVPreslection}. \Pquark can be \Pup, \Pdown, \Pstrange, \Pbottom or \Ptop. Unless specified, \Pquark, \Plepton and \Pnu represent either particles or the corresponding anti-particles. \Pphoton(BS) represents a real photon from beamstrahlung (BS). \Pphoton(EPA) represents a ``quasi-real'' photon, simulated with the Equivalent Photon Approximation.}
\label{tab:doubleHiggs3TeVPreslectionPart2}
\end{table}


%% Big appendixes should be split off into separate files, just like chapters
%\input{app-myreallybigappendix}
