\chapter{Introduction}
\label{chap:Introduction}

%% Restart the numbering to make sure that this is definitely page #1!
\pagenumbering{arabic}

%% Note that the citations in this chapter use the journal and
%% arXiv keys: I used the SLAC-SPIRES online BibTeX retriever
%% to build my bibliography. There are also quite a few non-standard
%% macros, which come from my personal collection. You can have them
%% if you want, or I might get round to properly releasing them at
%% some point myself.

\chapterquote{The journey of a thousand miles begins with a single step.}%
{Lao Zi, 604 BC $-$ 531 BC}%: Blackwood's Magazine May 1830

Since twenty years ago, the high energy physics community has been considering a next-generation electron$-$positron collider after the Large Hadron Collider (LHC). Measurements from the \LHC helped to establish the Standard Model of particle physics. Yet there are issues that the Standard Model could not explain. For example, the origin of the masses of neutrinos and the particles accounted for cosmic dark matter are questions that need to be addressed. Precision measurements from a next-generation electron$-$positron collider will hopefully provide answers to some of these questions.

The advantage of an \ee linear collider over a hadron collider include: few background events from photon$-$photon collisions; a similar rate  of pairs  production of all particles as photon couples to all particles equally; smaller  theoretical uncertainties on the electroweak interaction; and full reconstruction of events.

The International Linear Collider (\ILC)  \cite{Brau:2007zza}, and the Compact Linear Collider (\CLIC) \cite{Linssen:2012hp}, are two most promising candidates of the next-generation electron$-$positron collider. The \ILC is capable of operating at centre-of-mass energies from 250\,GeV to 500\,GeV. \CLIC can reach centre-of-mass energies from 350\,GeV to 3\,TeV. Both colliders will be able to measure Higgs couplings precisely via processes like \HepProcess{\ee \to \PZ \PHiggs} and measure top quark mass and couplings via processes such as \HepProcess{\ee \to \Ptop \APtop}.

The optimisation of the design of the detectors for the future linear colliders is crucial to improve the ability to reconstruct events. By reconstructing the event to the per-particle level, an event can be studied in detail. At the same time,  physics simulation studies are important to demonstrate the physics reach of the future linear collider.

 %. A better event reconstruction leads to  a precise measurement of the

%Future electron-positron linear colliders are capable of making precise measurements of the Higgs sector, as well as the top quark sector \cite{Brau:2007zza,Linssen:2012hp}. At a high centre-of-mass energy, the collider could search for new physics, such as supersymmertry particles, and  measure rare events, such as double Higgs production events. These measurements wold be difficult for the current proton-proton collider, limited by the underlying QCD interaction. Therefore, it is important to optimise the design of the future particle detector for the linear colliders to improve the event reconstruction and to perform physics simulation studies to demonstrate the superiority of the linear collider.



%Precision measurements will help us to understand Standard Model (SM) better. In the autumn of 2012, experiments in LHC discovered a particle consistent with being the SM Higgs boson \cite{Aad:2012tfa,Chatrchyan:2012ufa}. However, is it known that there are limitations to the capability of the hadron colliders to measure properties of colour-singlet scalar particles. The determination of the Higgs properties, whether it is a Standard Model Higgs, depends on the precise measurement on cross section of Higgs decay. At an electron-positron collider, it will be possible to measure many elementary particles to a high precision\cite{Abramowicz:2013tzc}, providing a probe to physics beyond standard model.



%Since the discovery of a particle consistent with the Standard Model Higgs boson at the LHC in 2012 \cite{Aad:2012tfa,Chatrchyan:2012ufa}, the natural step for high energy physicists is to understand the Higgs. Yet limited by the underlying QCD interaction from proton-anti-proton collision, one has great difficulties in measuring the properties of the Higgs precisely. However, next generation electron-positron linear colliders could make precise measurements in the Higgs sector, as well as the top quark sector \cite{Brau:2007zza,Linssen:2012hp}.

%This thesis contains the work  on the detector and the physics at future electron-positron linear colliders. Necessary background information is provided, followed by detailed discussions on three projects completed.

\CHAPTER{chap:Theory} starts with an overview of Standard Model of particle physics, including brief discussions on the quantum electrodynamics, quantum chromodynamics, and the elctroweak interaction. The focus of the Standard Model discussion is on the  Higgs mechanism and the Higgs boson in the Standard Model. The discussion then moves on to theories beyond the Standard Model, with an example of   a general parametrisation of the Higgs theory. The last part of the chapter is dedicated to the discussion on studying the correlation between the polarisations of the  tau pair from a boson decay to determine statistically if the parent boson is a  scalar or a vector.

 %identifying a Higgs boson from vector bosons using the tau pair decay channel.


In \Chapter{chap:Detector}, the detector designs currently considered for two future electron-positron linear colliders, the \ILC and \CLIC, are described. After a short introduction of the two colliders, the physics programme for these future colliders is discussed, followed by the impact of physics requirements on the detector design. Afterwards, the International Large Detector (\ILD), one detector option for the \ILC, is discussed in detail, followed by overviews on each sub-detector in the \ILD. The chapter finishes with a discussion on the modified \ILD detector concept for \CLIC.

In \Chapter{chap:Reconstruction}, softwares for event simulation, event reconstruction, and event analysis are  discussed.  \pandora, a world-leading pattern-recognition software for particle flow calorimetry, is presented. A discussion on jet algorithms is provided, followed by a discussion on  multivariate analysis, where different fitting models, optimisation, and overfitting are discussed. The multivariate analysis is used extensively in \Chapter{chap:Photon}, \Chapter{chap:Tau}, and \Chapter{chap:DoubleHiggs}.

%\pandora aims to reconstruct individual particles in the event.
 %The event reconstruction of future linear colliders share common software framework. Hence, the shared software for simulation and reconstruction is discussed first, with an emphasis on the \pandora, a world-leading pattern-recognition software for particle flow calorimetry. Some \CLIC specific issues are highlighted afterwards.


\CHAPTER{chap:Photon} describes several new \pandora algorithms for photon reconstruction. These algorithms reconstruct photons and address issues of photon fragments in the event reconstruction. After presenting the algorithms, performances of these algorithms are provided using photon samples and jet samples.

In \Chapter{chap:Tau}, a classification of tau lepton decay modes is presented. The analysis contains the event generation, simulation, reconstruction, and the use of the multivariate classifier for the classification. Utilising highly granular calorimeters, the resolutions of energy and invariant mass of the tau decay products are improved. A hypothesis test was performed for expected decay final states.  The performance of the tau decay mode classification is given, followed by an electromagnetic calorimeter optimisation study of the \ILD detector based on the tau decay mode classification. Lastly, the  tau decay mode classification is used in a proof-of-principle analysis to demonstrate the ability to use the tau pair polarisation correlation as a signature for Higgs boson using the tau pair decay process, where both \tauToPionBoth.

In \Chapter{chap:DoubleHiggs}, a full \CLICILD detector simulation study is performed for the double Higgs production channel, \eeToHH, via \WW fusion. Event generation and simulation will be discussed first. An overview of the analysis, including lepton finding and jet reconstruction, is presented, followed by an optimised multivariate analysis to distinguish signal from background processes. The optimised event selection is used to derive an estimate of the uncertainty on the cross section of double Higgs production at \CLIC. The event selection is further used to provide an estimate of the uncertainty on the measurements of  trilinear Higgs self coupling and quartic coupling at \CLIC.

%This thesis finishes with \Chapter{chap:summary}, where a summary is provided. 