\chapter{Introduction}
\label{chap:Introduction}

%% Restart the numbering to make sure that this is definitely page #1!
\pagenumbering{arabic}

%% Note that the citations in this chapter use the journal and
%% arXiv keys: I used the SLAC-SPIRES online BibTeX retriever
%% to build my bibliography. There are also quite a few non-standard
%% macros, which come from my personal collection. You can have them
%% if you want, or I might get round to properly releasing them at
%% some point myself.

\chapterquote{The journey of a thousand miles begins with a single step.}%
{Lao Zi, 604 BC - 531 BC}%: Blackwood's Magazine May 1830



Future electron-positron linear colliders are capable of making precise measurements of the Higgs sector, as well as the top quark sector \cite{Brau:2007zza,Linssen:2012hp}. At a high centre-of-mass energy, the collider could search for new physics such as supersymmertry particles, and  measure rare events, such as double Higgs production events. These measurements wold be difficult for the current proton-proton collider, limited by the underlying QCD interaction. Therefore, it is important to optimise the design of the future particle detector for the linear colliders to improve the event reconstruction and to perform physics simulation studies to demonstrate the superiority of the linear collider.


%This thesis contains the work  on the detector and the physics at future electron-positron linear colliders. Necessary background information is provided, followed by detailed discussions on three projects completed.

The thesis begins with overview of relevant theories on particle physics in \Chapter{chap:Theory}.  Firstly a brief review of the   current best particle theory, Standard Model of Particle Physics, is provided, including a short overview of the quantum electrodynamics, quantum chromodynamics, and the elctroweak interaction. The focus of the Standard Model discussion is on the  Higgs mechanism and the Higgs boson in Standard Model. The discussion then moves on to theories beyond the Standard Model, with an example of   a general parametrisation of the Higgs theory. The last part of the chapter dedicated to the discussion on identifying a Higgs boson from vector bosons using tau pair decay channel.


In \Chapter{chap:Detector}, the detector models used in the thesis are described in details. A general overview of two future electron-positron linear colliders, the International Linear Collider (\ILC) and the Compact Linear Collider (\CLIC), is provided. After a short discussion on the physics program for these future colliders, a discussion of the impact of physics and other requirements on the detector design is presented. Afterwards, the International Large Detector, one detector option for the International Linear Collider, is discussed in details, followed by overviews on each sub-detector in the International Large Detector. The chapter finishes with a discussion on the modified International Large Detector detector concept for the Compact Linear Collider, where the modifications of the detector are highlighted.

In the next chapter, \Chapter{chap:Reconstruction}, the software for event simulation and event reconstruction is  discussed, followed by a discussion on the analysis software. Future linear colliders share common software framework. Hence, shared software for simulation and reconstruction is discussed first, with an emphasis on the \pandora, a world-leading pattern-recognition software for particle flow calorimetry. Some \CLIC specific issues are highlighted afterwards. Analysis software, including jet algorithms, is presented. Lastly, the multivariate analysis is discussed in details, where different fitting models, optimisation, and overfitting are discussed.


\CHAPTER{chap:Photon} describes several \pandora algorithms regarding photon reconstruction. One algorithm performs the initial photon forming and photon ID test. Three algorithms are developed for the photon fragment removals. And one algorithm is developed to split the accidently merged photons. The core of identifying the photon is a two dimensional peaking finding algorithm. Having discussed the algorithms, performances of these algorithms are provided. Comparison with event reconstruction without photon reconstruction is also provided.

In \Chapter{chap:Tau}, a classification of the tau lepton decay modes is presented. The analysis contains the sample selection, pre-selection cuts, and the use of the multivariate classifier for the classification.  The performance of the tau decay mode classification will be given, followed by an \ECAL optimisation study using the tau decay mode classification. Lastly, the  tau decay mode classification is further used in a proof-of-principle analysis to demonstrate the ability to use the tau pair polarisation correlation as a signature for Higgs boson.


In \Chapter{chap:DoubleHiggs}, a full \CLICILD detector simulation study has been performed for the double Higgs production channel, \eeToHH, via \WW fusion. Event generation and simulation will be discussed first. An overview of the analysis, including lepton finding and jet reconstruction, is presented, followed by an optimised multivariate analysis to distinguish signal from background processes. The optimised event selection is used to derive an estimate of the uncertainty on the cross section of double Higgs production at the \CLIC. The event selection is further exploited to provide an estimate of the uncertainty on the measurements of  trilinear Higgs self coupling and quartic coupling at the \CLIC.

%This thesis finishes with \Chapter{chap:summary}, where a summary is provided. 